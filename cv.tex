\documentclass{customcv}

\author{John Doe}
\date{\today}
\title{Title}

\usepackage[ngerman]{babel}

\primary{DodgerBlue4}

\begin{document}

\makecontactheader{
  \location{location},
  \mail{mail},
  \phone{phone},
  \birthday{birthday},
  \linkedin{linkedin},
  \github{github}
}

\section{Education}

\eduentry{BS}
{University}
{course of study}
{Time Start - Time End}
{
  \begin{itemize}
    \item Some bullets you want to add here
    \item You could also just add plain text or any other stuff you want
  \end{itemize}
}

\section{Experience}

\expentry{Company}
{Title}
{Time Start - Time End}
{
  \smallskip
  Itemize or plain Text as you want.
  You can also customize the section names.
}

\section{Skills}

\entry{Keyword:}{List of Skills}


\section{Explaination}

The CV contains multiple different sections as shown in this example.
The section heading are customizable and the entrys can be used multiple times.

\subsection{Header}

The Header is created by the \texttt{\textbackslash{}makecontactheader} command.
That command takes a comma separated list to place under the authors name.
This list may contain Icons and links.
Some of them are already simplified by a command that has the same name as
displayed in the header of this document.
Those commands add an Icon and the link for the provided username if appropriate.

\subsection{EduEntry}

Usage:
\texttt{\textbackslash{}eduentry\{Degree\}\{University\}\{Course of study\}\{Dates\}\{Some Additional info\}}

The EduEntry makes a table with 3 columns.
The first contains the degree and the last the Dates.
The University is bold and the course of study follows.
The Additional Info can pe plain text or an itemize etc.

\subsection{ExpEntry}

Usage:
\texttt{\textbackslash{}expentry\{Company\}\{Title\}\{Dates\}\{Some Additional info\}}

The ExpEntry is the same as the EduEntry without the Degree Column.

\subsection{Entry}

Usage:
\texttt{\textbackslash{}entry\{Keyword\}\{Text\}}

Just a simple entry with a bold Keyword and normal Text.


\subsection{Color}

The primary color is set by \texttt{\textbackslash{}primary\{DodgerBlue4\}} and
expects a sting with an x11 name of a color (see the xcolor package documentation).



\end{document}
